\lab{LAB -- Center of Mass Velocity During an Elastic Collision}

\apparatus
\equip{Tracker video analysis software}
\equip{computer}

\longgoal
The purpose of this experiment is to measure the velocity of the center of mass of two pucks that make a collision on an air hockey table. You will measure the center-of-mass velocity before the collision and after the collision, and you will compare the results.

\introduction

The location of the center of mass of a system of two particles is

\begin{eqnarray*}
	\vec{r}_{cm}&=&\frac{m_1\vec{r}_1 + m_2\vec{r}_2}{m_1+m_2} \\
\end{eqnarray*}

This is a vector equation that must be true for both the x and y directions (for two dimensions).

\begin{eqnarray*}
	x_{cm} &=&\frac{m_1x_1 + m_2x_2}{m_1+m_2} \\
\end{eqnarray*}

\begin{eqnarray*}
	y_{cm} &=&\frac{m_1y_1 + m_2y_2}{m_1+m_2} \\
\end{eqnarray*}

Likewise, the center-of-mass velocity is

\begin{eqnarray*}
	\vec{v}_{cm}&=&\frac{m_1\vec{v}_1 + m_2\vec{v}_2}{m_1+m_2} \\
\end{eqnarray*}

Again, this equation must hold true for both the x and y components of the center-of-mass velocity.

\begin{eqnarray*}
	{v}_{cm,x}&=&\frac{m_1{v}_{1x} + m_2{v}_{2x}}{m_1+m_2} \\
\end{eqnarray*}

\begin{eqnarray*}
	{v}_{cm,y}&=&\frac{m_1{v}_{1y} + m_2{v}_{2y}}{m_1+m_2} \\
\end{eqnarray*}

\procedure

It is expected that you have completed the other video analysis experiments, so these instructions do not include details about how to use the \emph{Tracker} software.

\begin{enumerate}
	\item Download the video \file{collision-pucks.mov} from our course web site.
	\item Open \emph{Tracker} and insert the video.
	\item Play the video and view the motion of  the colliding pucks.
	\item Record the mass of each puck that is printed on the video's first frame.
	
	\tightframe{
	\bigskip
	$m_{blue} = $

	\bigskip
	$m_{red} = $

	\bigskip
	}
	
	\item Set the origin of your coordinate system. Any location is fine. I happened to choose the center of the blue puck in the first frame.
	\item Set the calibration using the meterstick on the left side of the video. You will have greater accuracy if you use 5 of the 10-cm segments for your calibration. In other words, stretch the calibration stick across 5 segments for a total length of 0.5 m.
	\item Mark the blue puck for each frame of the video.
	\item Mark the red puck for each frame of the video. (You will have to create another point mass first.)
	\item Using the graphs of $x$ vs. $t$ and $y$ vs. $t$ for each puck, measure the following quantities:
	
\begin{table}[htdp]
\caption{default}
\begin{center}
\begin{tabular}{|c|c|c|c|c|}
\hline
 & $v_{xi}$ & $v_{yi}$ & $v_{xf}$ & $v_{yf}$ \\
\hline
\hline
blue & & & & \\
\hline
red & & & & \\
\hline
\end{tabular}
\end{center}
\label{default}
\end{table}%

\end{enumerate}

\analysis

\smallframe{Calculate the x-component of the center-of-mass velocity \emph{before} the collision, ${v}_{cm,ix}$.}

\smallframe{Calculate the y-component of the center-of-mass velocity \emph{before} the collision, ${v}_{cm,iy}$.}

\smallframe{Calculate the x-component of the center-of-mass velocity \emph{after} the collision, ${v}_{cm,fx}$.}

\smallframe{Calculate the y-component of the center-of-mass velocity \emph{after} the collision, ${v}_{cm,fy}$.}

\smallframe{What is $\vec{v}_{cm,i}$? Write and sketch the vector.}

\smallframe{What is $\vec{v}_{cm,f}$? Write and sketch the vector.}

\emph{Tracker} can calculate and track the center of mass for you. The following steps will help you learn how to automatically calculate and tracker the center of mass.

\begin{enumerate}
	\item We need to define the masses of the pucks. Click the tab for \menu{mass A} in the Track Control toolbar. In the drop-down menu, select \menu{Define...}. In the resulting pop-up window, enter the mass of the puck for the parameter \menu{m} as shown in Figure \ref{video-center-of-mass/define-menu}.
	
\scaledimage{video-center-of-mass/define-menu}{Enter of the mass of the puck.}{0.4}

	\item Repeat the previous step for \menu{mass B} and enter its mass.

	\item  Click the \button{Create} button and select \menu{Center of Mass}, as shown in Figure \ref{video-center-of-mass/create-menu}.
	
\scaledimage{video-center-of-mass/create-menu}{Select \menu{Center of Mass} from the menu.}{0.5}

	\item You will see a new tab in the Track Control toolbar named {\bf cm}. Click \button{cm} to get the menu for the cm object shown in Figure \ref{video-center-of-mass/cm-menu}.   Click the \menu{Select Masses...} menu item.

\scaledimage{video-center-of-mass/cm-menu}{Click on \menu{Select Masses...} from the menu.}{0.5}

	\item An additional window will pop up. Select both masses ``mass A'' and ``mass B'' in this window and click \button{OK} as shown in Figure \ref{video-center-of-mass/select-masses-menu}.
	
\scaledimage{video-center-of-mass/select-masses-menu}{Check both masses (i.e. pucks) in this window.}{0.5}
		
	\item You will now see a track for the center of mass and you will see a graph of $x$ vs. $t$. 
	
\smallframe{By measuring the slope of $x$ vs. $t$ for the center of mass before the collision, what is $v_{cm,ix}$?}

\smallframe{By measuring the slope of $y$ vs. $t$ for the center of mass before the collision, what is $v_{cm,iy}$?}

\smallframe{By measuring the slope of $x$ vs. $t$ for the center of mass after the collision, what is $v_{cm,fx}$?}

\smallframe{By measuring the slope of $y$ vs. $t$ for the center of mass after the collision, what is $v_{cm,fy}$?}

\smallframe{How did the measurements for the center-of-mass velocity compare to what you calculated in the first part of the experiments?}

\smallframe{Did the collision significantly affect the center-of-mass velocity?}
		
\end{enumerate}

\report

%\begin{enumerate}
%	\item Is the center-of-mass velocity constant or non-constant?
%	\item What is the net force on the system of pucks?
%	\item The collision occurs in approximately 1/30 of a second. What is the force by the red puck on the blue puck during the collision? 
%	\item What is the force by the blue puck on the red puck during the collision?
%\end{enumerate}

\begin{description}

\item[C]  Complete the experiment and report your answers for the following questions. Please type your report.

\begin{enumerate}
	 \item Using data from Table 1, what is the center-of-mass velocity vector before the collision?
	 \item Using data from Table 1, what is the center-of-mass velocity vector after the collision?
	 \item What is the center-of-mass velocity before the collision, as measured by the slope of the $x$ and $y$ vs. time graphs of the center of mass? Report your answer as a vector.
	 \item What is the center-of-mass velocity after the collision, as measured by the slope of the $x$ and $y$ vs. time graphs of the center of mass? Report your answer as a vector.
\end{enumerate}

\item[B] Do all parts for {\bf C} and answer the following questions.

\begin{enumerate}
	\item Can you conclude that the center-of-mass velocity is constant (or nearly constant) during the collision?
	\item You used Tracker to calculate the location of the center of mass and mark it on the video.  What does the fact that it is a straight line (both before and after the collision) and the fact that the spacing between marks is uniform (i.e. the same) tell you about the center of mass velocity?
	\item What is the acceleration of the center of mass?
	\item What is the net force on the center of mass?  (Hint: use Newton's second law in your reasoning.)
\end{enumerate}

\item[A] Do all parts for {\bf B} and answer the following questions.

\begin{enumerate}
	\item The collision occurs in approximately 1/30 of a second, so $\Delta t=1/30\ \second$. Use Newton's second law and the data for $v_{xi}$ , $v_{yi}$ , $v_{xf}$ , and $v_{yf}$ for the blue puck to calculate the force on the blue puck during the collision. Express your answer as a vector.
	\item Use data for the red puck to calculate the force on the red puck during the collision. Express your answer as a vector.
	\item Compare the force on the blue puck and the force on the red puck during the collision. What do you notice?  (This is a VERY big observation. It is so important that it is called \emph{Newton's Third Law} and governs objects that interact via electric or gravitational forces).
\end{enumerate}
\end{description}

% Answers

%[C]
%
%1. vi = (0.290, 0.245) m/s
%2. vf = (0.290, 0.242) m/s
%3. vcmi = (0.287, 0.242) m/s
%4. vcmf = (0.287, 0.242) m/s
%
%[B]
%
%1.  Yes, cm v is constant.
%2.  The equal spacing between marks tells you that the cm velocity is constant.
%3. a = 0
%4. Fnet =0
%
%[A]
%
%1. Fnetx on blue puck = -0.639 N; Fnety on blue puck = 0.667 N
%2. Fnetx on red puck = 0.639 N; Fnety on red puck = 0.674 N
%3.  The net force on the red puck is the opposite of the net force on the blue puck. The sum of these two forces is zero.

