\lab{GAME -- Asteroids}

\apparatus
\equip{Computer}
\equip{VPython -- www.vpython.org}

\longgoal

The purpose of this activity is to modify an asteroids game so that when a large asteroid breaks into two smaller asteroids, the center of mass continues with the same velocity, as you observed in the experiment of the colliding pucks.

\procedure


\begin{enumerate}

\subsection*{Playing Asteroids}

\item Play the game \emph{Asteroids}.  A link is available from our course web site. Note the actions of the up, down, left, and right keystrokes and how they affect the motion of the spaceship.

\item Download the file \file{asteroids-template.py} from our course web site, and open the file in VPython.

\item Run the program. Note how the up, down, left, and right keystrokes affect the motion of the spaceship. It's different than in the other Asteroids game. Also, note that hitting an asteroid with a bullet does not make it break up into pieces. The program is printed below for reference.

\begin{vpythonblock}
from visual import *
import random

#converts an angle in degrees to an angle in radians
def rad(degrees): 
    radians=degrees*pi/180
    return radians

#pause and wait for mouse or keyboard event, then continue
def pause():
    while True:
        rate(50)
        if scene.mouse.events:
            m = scene.mouse.getevent()
            if m.click == 'left': return
        elif scene.kb.keys:
            k = scene.kb.getkey()
            return

#checks for a collision between two spheres
def collisionSpheres(sphere1, sphere2):
    dist=mag(sphere1.pos-sphere2.pos)
    if(dist<sphere1.radius+sphere2.radius):
        return True
    else:
        return False

#checks for a collision between a cone and a sphere
def collisionConeSphere(c, s):

    #result is the variable that we will return
    #default is False
    result=False

    #check pos of cone
    if(collisionSphereAndPoint(s,c.pos)):
        result=True
    #check tip of cone
    result=False
    tip=c.pos+c.axis
    if(collisionSphereAndPoint(s,tip)):
        result=True
    #check edge of radius in x-y plane 1
    r1=c.radius*cross(vector(0,0,1),norm(c.axis))
    if(collisionSphereAndPoint(s,r1+c.pos)):
        result=True
    #check edge of radius in x-y plane 2
    r2=-c.radius*cross(vector(0,0,1),norm(c.axis))
    if(collisionSphereAndPoint(s,r2+c.pos)):
        result=True

    #return result
    return result

#determines whether a point is within a sphere or not
#returns boolean
def collisionSphereAndPoint(sphereObj, targetVector):
    dist=mag(sphereObj.pos-targetVector)
    if(dist<sphereObj.radius):
        return True
    else:
        return False
    
#creates four asteroids, one on each side of the scene
def createAsteroids():

    #asteroid comes from the right
    asteroid=sphere(pos=vector(20,0,0), radius=1, color=color.cyan)
    asteroid.pos.y=random.randrange(-20,20,5)
    asteroid.m=1
    asteroid.v=vector(0,0,0)
    asteroid.v.x=-random.randint(1,5)
    asteroid.v.y=random.choice((1,-1))*random.randint(1,5)
    asteroidList.append(asteroid)

    #asteroid comes from the left
    asteroid=sphere(pos=vector(-20,0,0), radius=1, color=color.cyan)
    asteroid.pos.y=random.randrange(-20,20,5)
    asteroid.m=1
    asteroid.v=vector(0,0,0)
    asteroid.v.x=random.randint(1,5)
    asteroid.v.y=random.choice((1,-1))*random.randint(1,5)
    asteroidList.append(asteroid)

    #asteroid comes from the top
    asteroid=sphere(pos=vector(0,20,0), radius=1, color=color.cyan)
    asteroid.pos.x=random.randrange(-20,20,5)
    asteroid.m=1
    asteroid.v=vector(0,0,0)
    asteroid.v.x=random.choice((1,-1))*random.randint(1,5)
    asteroid.v.y=-random.randint(1,5)
    asteroidList.append(asteroid)

    #asteroid comes from the bottom
    asteroid=sphere(pos=vector(0,-20,0), radius=1, color=color.cyan)
    asteroid.pos.x=random.randrange(-20,20,5)
    asteroid.m=1
    asteroid.v=vector(0,0,0)
    asteroid.v.x=random.choice((1,-1))*random.randint(1,5)
    asteroid.v.y=random.randint(1,5)
    asteroidList.append(asteroid)

#scene size
scene.range=20
scene.width=700
scene.height=700

#create the spaceship as a cone
spaceship = cone(pos=(0,0,0), axis=(2,0,0), radius=1, color=color.white)
fire = cone(pos=(0,0,0), axis=-spaceship.axis/2, radius=spaceship.radius/2, color=color.orange)

#initial values for mass, velocity, thrust, and net force
spaceship.m=1
spaceship.v=vector(0,0,0)
thrust=0
Fnet=vector(0,0,0)

#bullets
bulletspeed=10
bulletsList=[]

#angle to rotate
dtheta=rad(10)

#clock
t=0
dt=0.005

#asteroids
Nleft=0 #counter for number of asteroids left in the scene
asteroidList=[]
createAsteroids()

while spaceship.visible==1:
    rate(200)

    if scene.kb.keys:
        k = scene.kb.getkey()
        if k == "up": #turn thruster on
            thrust=6
        elif k=="left": #rotate left
            spaceship.rotate(angle=-dtheta, axis=(0,0,-1));
        elif k=="right": #rotate right
            spaceship.rotate(angle=dtheta, axis=(0,0,-1));
        elif k==" ": #fire a bullet
            bullet=sphere(pos=spaceship.pos+spaceship.axis, radius=0.1, color=color.yellow)
            bullet.v=bulletspeed*norm(spaceship.axis)+spaceship.v
            bulletsList.append(bullet)
        elif k=="q": #pause the game
            pause()
        else: #turn thruster off
            thrust=0

    Fnet=thrust*norm(spaceship.axis)
    spaceship.v=spaceship.v+Fnet/spaceship.m*dt
    spaceship.pos=spaceship.pos+spaceship.v*dt
    fire.pos=spaceship.pos
    fire.axis=-spaceship.axis/2


    #check if the spaceship goes off screen and wrap
    if spaceship.pos.x>20 or spaceship.pos.x<-20:
        spaceship.pos=spaceship.pos-spaceship.v*dt
        spaceship.pos.x=-spaceship.pos.x
    if spaceship.pos.y>20 or spaceship.pos.y<-20:
        spaceship.pos=spaceship.pos-spaceship.v*dt
        spaceship.pos.y=-spaceship.pos.y

    #update positions of bullets and check if bullets go off screen
    for thisbullet in bulletsList:
        if thisbullet.pos.x>20 or thisbullet.pos.x<-20:
            thisbullet.visible=0
        if thisbullet.pos.y>20 or thisbullet.pos.y<-20:
            thisbullet.visible=0
        if thisbullet.visible != 0:
            thisbullet.pos=thisbullet.pos+thisbullet.v*dt

    #update positions of asteroids
    for thisasteroid in asteroidList:
        if thisasteroid.visible==1:
            thisasteroid.pos=thisasteroid.pos+thisasteroid.v*dt
            #check for collision with spaceship
            if(collisionConeSphere(spaceship,thisasteroid)):
                spaceship.visible=0
                fire.visible=0
            #wrap at edge of screen
            if thisasteroid.pos.x>20 or thisasteroid.pos.x<-20:
                thisasteroid.pos=thisasteroid.pos-thisasteroid.v*dt
                thisasteroid.pos.x=-thisasteroid.pos.x
            if thisasteroid.pos.y>20 or thisasteroid.pos.y<-20:
                thisasteroid.pos=thisasteroid.pos-thisasteroid.v*dt
                thisasteroid.pos.y=-thisasteroid.pos.y
            #check for collision with bullets
            for thisbullet in bulletsList:
                if(collisionSpheres(thisbullet,thisasteroid)and thisbullet.visible==1):
                   thisasteroid.visible=0
                   thisbullet.visible=0

    Nleft=0 #have to reset this before counting
    for thisasteroid in asteroidList:
        if thisasteroid.visible:
            Nleft=Nleft+1

    #create more asteroids if all are gone
    if Nleft==0:
        createAsteroids()

    #update fire
    if thrust==0:
        fire.visible=0
    else:
        fire.visible=1

    t=t+dt
\end{vpythonblock}

\item Answer the following questions.

\begin{enumerate}
	\item What is the magnitude of the thrust of the engine when it is firing?
	\item What line number calculates the net force on the spaceship?
	\item What line numbers update the velocity and position of the spaceship?
	\item What line numbers update the positions of the asteroids?
	\item What line numbers update the positions of the bullets?
	\item What line number makes the asteroid disappear when it is hit by a bullet?
	\item How many asteroids are created when the function \code{createAsteroids()} is called and where do these asteroids come from?
	\item The velocities of the asteroids are randomized. What is the maximum possible velocity of an asteroid? What is the minimum possible velocity of an asteroid?  (Note that the directions are randomized as well. By ``maximum velocity'' I am referring to the maximum absolute value of the x and y components of the velocity vector. By ``minimum velocity'', I am referring to the minimum absolute value of the x and y components of the velocity vector.)
	\item If you decrease the mass of the spaceship (i.e. change it from 1 to a smaller number like 0.5), how would it affect the spaceship's motion when the engine is firing? Answer the question, then test your answer to see if it is correct, and then comment on whether your observation matched your prediction.
	
	\smallframe{\ }
	
\end{enumerate}

\subsection*{Adding Asteroids Fragments}

In the last chapter, you learned that the center of mass of a two-body system is constant. Whether it is a collision of two objects or an exploding fireworks shell, during the collision or explosion, the center of mass of the system is the same.

In the classic Asteroids game, when a large asteroid is shot with a bullet, it breaks into two fragments. Since the center of mass of the system must remain constant then

\begin{eqnarray*}
	\vec{v}_{ast} & = & \frac{m_1\vec{v}_1+m_2\vec{v}_2}{m_1+m_2} \\
\end{eqnarray*}

Define the total mass of the fragments as $M=m_1+m_2$, which is the mass of the asteroid before the explosion. Then,

\begin{eqnarray*}
	\vec{v}_{ast} & = & \frac{m_1\vec{v}_1+m_2\vec{v}_2}{M} \\
\end{eqnarray*}

In our game, let's assume that the asteroid breaks into two equal mass fragments that are each $1/2$ the total mass of the asteroid. Then, $m_1=m_2=1/2M$. Thus,

\begin{eqnarray*}
	\vec{v}_{ast} & = & \frac{1/2M\vec{v}_1+1/2M\vec{v}_2}{M} \\
	\vec{v}_{ast} & = &  \frac{\vec{v}_1+\vec{v}_2}{2}
\end{eqnarray*}

In other words, since the fragments have equal masses, then the asteroid's velocity is the arithmetic mean of the velocities of the fragments (i.e. the sum of their velocities divided by 2). 

In our game, we will randomly assign the velocity of fragment 1 so that it will shoot off with a random speed in a random direction. Then we will calculate the velocity of fragment 2 so that

\begin{eqnarray*}
	\vec{v}_{2} & = & 2\vec{v}_{ast}-\vec{v}_1\\
\end{eqnarray*}

\item Above the \code{while} loop where the asteroidList is created, you need to create a list called \code{fragmentList}. To do this, type the following line:

\begin{myvpython}
fragmentList=[]
\end{myvpython}

\item Create a function that will create the fragments when an asteroid is hit. Type the following code and replace the commented line with the correct calculation for the velocity of fragment 2.

\begin{myvpython}
def createFragments(asteroid):
    fragment1=sphere(pos=asteroid.pos, radius=0.5, color=color.magenta)
    fragment2=sphere(pos=asteroid.pos, radius=0.5, color=color.magenta)
    fragment1.m=0.5
    fragment2.m=0.5
    fragment1.v=vector(0,0,0)
    fragment1.v.x=random.choice((1,-1))*random.randint(1,5)
    fragment1.v.y=random.choice((1,-1))*random.randint(1,5)
#    fragment2.v=
    fragmentList.append(fragment1)
    fragmentList.append(fragment2)
\end{myvpython}

\item In the block of code that checks for a collision between a bullet and asteroid, call the \code{createFragments} function using:

\begin{myvpython}
                   createFragments(thisasteroid)
\end{myvpython}

\item You will need to update the positions of the fragments (i.e. make them move). It's easiest to copy and paste the section of code for the asteroids and change the variables names to the appropriate names for the fragments. Here's an example of what it should look like:

\begin{myvpython}
    #update positions of fragments
    for thisfragment in fragmentList:
        if thisfragment.visible==1:
            thisfragment.pos=thisfragment.pos+thisfragment.v*dt
            #check for collision with spaceship
            if(collisionConeSphere(spaceship,thisfragment)):
                spaceship.visible=0
                fire.visible=0
            #wrap at edge of screen
            if thisfragment.pos.x>20 or thisfragment.pos.x<-20:
                thisfragment.pos=thisfragment.pos-thisfragment.v*dt
                thisfragment.pos.x=-thisfragment.pos.x
            if thisfragment.pos.y>20 or thisfragment.pos.y<-20:
                thisfragment.pos=thisfragment.pos-thisfragment.v*dt
                thisfragment.pos.y=-thisfragment.pos.y
            #check for collision with bullets
            for thisbullet in bulletsList:
                if(collisionSpheres(thisbullet,thisfragment)and thisbullet.visible==1):
                   thisfragment.visible=0
                   thisbullet.visible=0
\end{myvpython}

\item The section of code that counts how many asteroids are left also needs to count how many fragments are left. After the \code{for} loop that counts the asteroids that are left, add a second for \code{loop} that counts the remaining fragments.

\begin{myvpython}
    for thisfragment in fragmentList:
        if thisfragment.visible:
            Nleft=Nleft+1
\end{myvpython}

\item That should do it. Run your code and fix any errors.

\item Note that we could have written our code more efficiently. Whenever you find that you are copying and pasting the same code over and over, you might want to consider putting it into a function and calling the function. For example, we have to wrap the motion of the spaceship, asteroids, and fragments. We can create a \code{wrap()} function that takes an argument of the object (like spaceship) and checks to see if it is off screen. If it is off screen, it wraps the position of the object. If you notice ways like this to reduce the number of lines of code, feel free to make those changes.

\end{enumerate}

\pagebreak

\analysis

\begin{description}

\item[C] Complete this exercise and do the following.

\begin{enumerate}
	\item Add a point system that tallies points whenever an asteroid is shot.
	\item Add a keystroke for the ``down'' arrow key that sets the thrust to a negative value (such as \code{thrust=-6}).
	\item Print the total number of points at the end of the game when the spaceship collides with an asteroid.
\end{enumerate}


\item[B] Do everything for {\bf C} and the following.

\begin{enumerate}
	\item Place the entire \code{while spaceship.visible==1:} loop inside another \code{while 1:} loop so that after the spaceship collides with an asteroid the program will not end, but rather it will begin again.
	\item You will need to reinitialize your variables for the velocity, thrust, and net force on the spaceship, along with your lists, in order to reset the game. You can copy code like the following and paste it just after the \code{while 1:} statement.
	
\begin{myvpython}
spaceship.visible=1
spaceship.v=vector(0,0,0)
thrust=0
Fnet=vector(0,0,0)

#bullets
bulletsList=[]

#asteroids
Nleft=0 #counter for number of asteroids left in the scene
asteroidList=[]
createAsteroids()

#fragments
fragmentList=[]
\end{myvpython}

	\item Call the \code{pause()} function after resetting your variables and lists and before the \code{while spaceship.visible==1:} statement.
	
	\item Allow a total of 3 lifetimes for the spaceship and end the program after the third one.

\end{enumerate}

\item[A] Do everything for {\bf B} with the following modifications and additions.

\begin{enumerate}
	\item Add fragmentation for the asteroids so that the physics is correct and all fragments have to be shot before the game is reset.
\end{enumerate}




\end{description}

