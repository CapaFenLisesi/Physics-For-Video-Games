\myappendix{functionsandgraphs}{Functions and Graphs}

When making a graph, you generally plot the independent variable on the x-axis and the dependent variable on the y-axis as shown in Figure \ref{appendix-functions-and-graphs/y-x}.  Since $y$ depends on x, then we can determine the functional relationship, $y=f(x)$ by analyzing the graph.

\scaledimage{appendix-functions-and-graphs/y-x}{Dependent variable on the $y$ axis and independent variable on the $x$ axis.}{0.8}

There are many relationships between $y$ and $x$ that are common in physics.  Here are a few typical ones.

\subsection*{y is directly proportional to x}

This relationship is described by the equation

\begin{eqnarray*}
	y & = & mx \\
\end{eqnarray*}

where $m$ is the slope of the graph. The result is a straight line of non-zero slope that passes through the origin, so that $y=$ at $x=0$, as shown in Figure \ref{appendix-functions-and-graphs/direct}.

\scaledimage{appendix-functions-and-graphs/direct}{$y$ is directly proportional to $x$}{0.8}

\subsection*{y is linearly related to x}

This relationship is described by the equation

\begin{eqnarray*}
	 y & = &  mx+b
\end{eqnarray*}	 

where $m$ is the slope and $b$ is the y-intercept.  The resulting graph is a line of constant, non-zero slope.  Note that the slope is not necessarily positive and the y-intercept is not necessarily zero.  In the graph in Figure \ref{appendix-functions-and-graphs/linear}, the slope is negative and the y-intercept is equal to 3.

\scaledimage{appendix-functions-and-graphs/linear}{y is linearly related to x.}{0.8}
 
\subsection*{y is proportional to the square root of x}

When $y^2$ is proportional to x, the graph is a side-opening parabola.  The equation that describes this relationship is

\begin{eqnarray*}
	 y^2 & = & Cx
\end{eqnarray*}	 
 
and the graph looks something like the one shown in Figure \ref{appendix-functions-and-graphs/y2-x}. Taking the square root of Eq. 3  and setting the constant $\sqrt{C}$ to the letter $A$ shows that $y$ is proportional to $\sqrt{x}$.

\begin{eqnarray*}
	 y& = & A\sqrt{x}
\end{eqnarray*}	

\scaledimage{appendix-functions-and-graphs/y2-x}{$y$ is proportional to $\sqrt{x}$}{0.8}


\subsection*{quadratic}

This relationship is a top-opening parabola like graph shown in Figure \ref{appendix-functions-and-graphs/y-x2}.


\scaledimage{appendix-functions-and-graphs/y-x2}{quadratic}{0.8}

 
The equation that describes this relationship is

\begin{eqnarray*}
	 y & = &  ax^2+bx+c
\end{eqnarray*}	 

The constant $a$ determines the value of $y$ at $x =0$. The constant $b$ determines the slope of a tangent line (i.e. line tangent to the function) at $x=0$. The constant $c$ determines the concavity of the parabola.  If $c$ is positive, the function has an upward concavity. If $c$ is negative, then the function has a downward concavity.  Can you figure out what the signs are for the constants $a$, $b$, and $c$ in the previous graph?

\subsection*{$y$ is inversely proportional to x}

For this relationship, as $x$ gets larger, $y$ gets smaller.  As $x$ gets smaller, $y$ gets larger.  The equation relating $y$ and $x$ is of the form:

\begin{eqnarray*}
	 y & = &  \frac{A}{x}
\end{eqnarray*}	 

This curve assymptically approaches $y=0$ as $x\to\infty$.  Also, notice that as $x$ approaches zero, $y$ becomes infinite. An example graph of $y$ vs. $x$ for an inverse relationship is shown in Figure \ref{appendix-functions-and-graphs/inverse}.

\scaledimage{appendix-functions-and-graphs/inverse}{$y$ is inversely proportional to $x$}{0.8}

\newpage

\subsection*{$y$ is proportional to the cosine (or sine) of x}

For this relationship, $y$ ``oscillates."  The equation relating $y$ and $x$ is

\begin{eqnarray*}
	 y & = &  Acos(\omega x + \phi) + C
\end{eqnarray*}	 

This relationship is referred to as \emph{sinusoidal} regardless of whether you choose to use the cosine or sine function. The constant, $A$, is called the amplitude.  $\omega$ is called the angular frequency. $\phi$ is called the phase.  Whether the equation is written with a cosine function or with a sine function is unimportant. It simply alters the value of the phase. The constant, $C$, shifts the entire function upward or downward on the graph.

An example graph of $y$ vs. $x$ for a sinusoidal relationship is shown in Figure \ref{appendix-functions-and-graphs/sine}.

\scaledimage{appendix-functions-and-graphs/sine}{$y$ is sinusoidal}{0.8}


\subsection*{No relationship between $y$ and $x$}

If there is no relationship between $y$ and x, then changing $x$ will not result in a change in $y$.  In other words, $y$ is constant, regardless of the value of $x$.  What does a graph look like if $y$ is constant?  It is a straight line of zero slope as shown in Figure \ref{appendix-functions-and-graphs/no-relationship}.


\scaledimage{appendix-functions-and-graphs/no-relationship}{$y$ is not related to $x$}{0.8}

