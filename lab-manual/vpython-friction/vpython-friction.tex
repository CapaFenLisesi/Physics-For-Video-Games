
\lab{PROGRAM -- Modeling motion with friction}

\apparatus

\equip{VPython}
\equip{computer}

\longgoal

In this activity, you will learn how to add a frictional force to a simulation.
\introduction

We are going to model the motion of a puck in air hockey that is slowed by a frictional force between the puck and the table. Though in practice, a puck in air hockey may be more influenced by air drag than friction with the table, we will treat the frictional force as \emph{sliding friction}.  Sliding friction occurs when one objects slides against the surface of another object. In the simplest situations, slide friction is:

\begin{eqnarray*}
	\vec{F}_{slide} & = & -\mu_k F_{\perp}\hat{v} \\
\end{eqnarray*}

where $\mu_k$ is the coefficient of kinetic friction, $F_{\perp}$ is the perpendicular component of the contact force, and $\hat{v}$ is the unit vector pointing in the direction of the velocity of the object. In the case of a puck sliding on a level air hockey table, $F_{\perp}=mg$, the weight of the puck. Notice the negative sign. This means that sliding friction is always opposite the velocity of the object relative to the surface it is in contact with.

The coefficient of kinetic friction $\mu_k$ is a constant that depends on the materials in contact. For example, wood sliding on glass, wood sliding on concrete, and wood sliding on sand paper have different values of $\mu_k$. A higher coefficient of friction means that there is a greater frictional force. Smaller coefficient of friction is less friction, and zero coefficient of friction is what we call ``frictionless'' (which doesn't exist in practice though friction might be negligible if it is small enough).

\procedure

To practice modeling motion with friction, we will create a one-dimensional golf putt with constant ``sliding'' friction. (Yes, the ball is rolling, but rolling friction can be characterized as $\vec{F}_{roll} = -\mu_{roll} mg\hat{v}$.)

\begin{enumerate}

\item Begin by typing the following template for a golf ball rolling in the x-direction on a green.

\begin{vpythonblock}
from visual import *

scene.range=20
scene.width=700
scene.height=700

ground = box(pos=vector(0,0,0), size=(40,40,1), color=color.green)
ball = sphere(pos=vector(-18,0,0), radius=0.5, color=color.white)
ground.pos.z=ground.pos.z-ground.width/2-ball.radius
hole = cylinder(pos=(15,0,ground.pos.z+ground.width/2),axis=vector(0,0,1), radius=3*ball.radius, color=(0.8,0.8,0.8))
hole.pos.z=hole.pos.z-mag(hole.axis)*0.9
putt = arrow(pos=ball.pos, axis=(0,0,0), shaftwidth=0.5, color=color.yellow)

#ball, friction, and grav
ball.m=0.045
g=10
mu=0.1

#speed
initialspeed=5

#velocity vector
scale=initialspeed
putt.axis=scale*vector(1,0,0)
ball.v=initialspeed*vector(1,0,0)


#clock
dt=0.01
t=0

scene.mouse.getclick()

while 1:
        rate(100)

        vhat=ball.v/mag(ball.v)
        Fnet=-mu*ball.m*g*vhat
#        ball.v=
#        ball.pos=
        putt.pos=ball.pos

        scale=mag(ball.v)
        putt.axis=scale*vector(1,0,0)

        t=t+dt

\end{vpythonblock}

\item Fill in the commented lines and run the program.

\item Change the initial speed until you can get the ball to stop in the hole.

\item Now change the coefficient of friction to either a smaller or larger value and find the new initial speed needed to get the ball into the cup.

\end{enumerate}

%To practice modeling motion with friction, we will create a one-player air hockey game with elastic collisions between the puck and the walls. The goal is to keep the puck out of your own goal. 


\report

\begin{description}

\item[C]  Complete the experiment and report your answers for the following questions.

\begin{enumerate}
	 \item Place the cup at another location that is not on the $+x$ axis, such as the top corner of the green.
	 \item Find the initial velocity vector needed to get the ball into the cup, if $mu=0.1$.
\end{enumerate}

\item[B] Do all parts for {\bf C} do the following.

\begin{enumerate}
 \item Add keyboard interaction so that by using the up and down arrows, you can change the initial speed of the ball before you putt the ball. Hitting the space bar actually putts the ball. It might be a good idea to put a different while loop with the keyboard interactions before the while loop for the motion of the ball. Use your tank wars program for a reminder on how to write the code. Use the initial position of the hole at $x=15$ ft, along the x-axis in order to test out your code.
 \item Alert the player if the ball goes into the hole.
 \item Alert the player if the ball goes off screen.
\end{enumerate}

\item[A] Do all parts for {\bf C} and {\bf B} and do the following.

\begin{enumerate}
 \item Add keyboard interactions that allow the user to change the angle of the initial velocity of the ball.
 \item Place the hole at a location not on the $x$ axis and allow the user to change both the initial speed and direction of the putt in order to get it into the hole.
\end{enumerate}
\end{description}