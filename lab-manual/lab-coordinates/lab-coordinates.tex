\lab{LAB:  Coordinates}

\apparatus
\equip{meterstick or tape measure}
\equip{graph paper}

\longgoal

To measure the coordinates of objects in the classroom.

\procedure

\begin{enumerate}
	\item Define the origin of our coordinate system to be at the left corner of the front of the room (if you are facing toward the front), at the floor.
	\item Define the $+x$ direction to be to the right, if facing the front of the room.
	\item Define the $+y$ direction to be upward toward the ceiling.
	\item Define the $+z$ direction to be along the side wall toward the back of the room.
	\item Using a meterstick, determine the coordinates of the center of the seat of your chair.
\end{enumerate}

\extension

\begin{description}
	\item [C] What are the coordinates of the center of your seat?
	\item [B] (Do C first.) Using graph paper, draw an x-z coordinate system with the origin in the top-left corner of the paper. (Note: it's ok to use your paper in landscape or portrait orientation depending on what is most convenient. You are sketching a top view of the room. Assume that the front wall with the whiteboard is at the top edge of the paper.) Draw the seat of your chair (as a circle) at the correct location on the coordinate system. Assume that the chair is underneath the table. Be sure to indicate the scale of your coordinate system by labeling the axes with both numbers and units. Sketch the boundaries of the room. Select the values of your gridlines so that the coordinate system takes up as much of the paper as possible.
	\item [A]  (Do B and C first.) Determine the coordinates of the center of your table in the room. Measure the length and width of the table and accurately sketch the boundary of your table on the coordinate system.
\end{description}
